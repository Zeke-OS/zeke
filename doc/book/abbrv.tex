\section*{List of Abbreviations and Meanings}

\begin{acronym}[UniversalSpacer]
\acro{POSIX}[POSIX]{Portable Operating System Interface for uniX\acroextra{ is a
  family of standards specifying a standard set of features for compatibility
  between operating systems.}}
\acro{HAL}[HAL]{Hardware Abstraction Layer\acroextra{.}}

\acro{inode}[inode]{\acroextra{inode is the file index node in Unix-style file
  systems that is used to represent a file file system object.}}
\acro{lfn}[LFN]{Long File Name\acroextra{.}}
\acro{mbr}[MBR]{Master Boot Record\acroextra{.}}
\acro{ramfs}[ramfs]{RAM file system\acroextra{.}}
\acro{sfn}[SFN]{Short File Name\acroextra{.}}
\acro{vfs}[VFS]{Virtual File System\acroextra{.}}
\acro{vnode}[vnode]{\acroextra{vnode is like inode but it's used as an
  abstraction level for compatibility between different file systems in Zeke.
  Particularly it's used at VFS level.}}

\acro{bio}[bio]{IO Buffer Cache\acroextra{.}}
\acro{dynmem}[dynmem]{(dynamic memory)\acroextra{ is the block memory allocator
  system in Zeke, allocating in 1 MB blocks.}}
\acro{vralloc}[VRAlloc]{Virtual Region Allocator\acroextra{.}}

\acro{MIB}[MIB]{Management Information Base\acroextra{ tree.}}

\acro{ipc}[IPC]{Inter-Process Communication\acroextra{.}}
\acro{shmem}[shmem]{Shared Memory\acroextra{.}}
\acro{rcu}[RCU]{Read-Copy-Update\acroextra{.}}

\acro{elf}[ELF]{Executable and Linkable
  \acroextra{ or Extensible Linking Format.}}

\acro{PUnit}[PUnit]{a portable unit testing framework for C\acroextra{.}}
\acro{GCC}[GCC]{GNU Compiler Collection\acroextra{.}}
\acro{GLIBC}[GLIBC]{The GNU C Library\acroextra{.}}
\acro{CPU}[CPU]{Central Processing Unit\acroextra{ is, generally speaking, the
  main processor of some computer or embedded system.}}

\acro{DMA}[DMA]{Direct Memory Access\acroextra{ is a feature that allows
 hardware subsystems to commit memory transactions without \acsu{CPU}'s
 intervention.}}
\acro{PC}[PC]{Program Counter\acroextra{ register .}}
\end{acronym}
