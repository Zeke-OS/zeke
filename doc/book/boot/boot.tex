\part{Bootstrapping}
\chapter{Bootloader}
\chapter{Kernel Initialization}

\section{Introduction}

Kernel initialization order is defined as follows:
\begin{itemize}
    \item \verb+hw_preinit+
    \item constructors
    \item \verb+hw_postinit+
    \item \verb+kinit()+
    \item uinit thread
\end{itemize}

After kinit, scheduler will kick in and initialization continues in
\verb+sinit(8)+ process that is bootstrapped by uinit thread.

\section{Kernel module/subsystem initializers}

There are currently four types of initializers supported:

\begin{itemize}
    \item \textbf{hw\_preinit} for mainly hardware related initializers
    \item \textbf{hw\_postinit} for hardware related initializers
    \item \textbf{constructor} (or init) for generic initialization
\end{itemize}

Every initializer function should contain \verb+SUBSYS_INIT("XXXX init");+
before any functional code and optionally before that a single or multiple
\verb+SUBSYS_DEP(YYYY_init);+ lines declaring subsystem initialization
dependencies.

Descturctors are not currently supported in Zeke but if there ever will be LKM
support the destructors will be called when unloading the module.

Listing \ref{list:kmodinit} shows the constructor/intializer notation used by
Zeke subsystems. The initializer function can return a negative errno code to
indicate an error, but \verb+-EAGAIN+ is reserved to indicate that the
initializer was already executed.

\lstinputlisting[label=list:kmodinit,%
caption=kmod.c]{boot/kmod.c}

Constructor prioritizing is not supported and \verb+SUBSYS_DEP+ should be used
instead to indicate initialization dependecies.

hw\_preinit and hw\_postinit can be used by including \verb+kinit.h+ header file
and using the notation as shown in \ref{list:hwprepostinit}. These should be
rarely needed and used since preinit doesn't allow many kernel features to be
used and postinit can be called after the scheduler is already running.

\lstinputlisting[label=list:hwprepostinit,%
caption=hwprepostinit.c]{boot/hwprepostinit.c}


\chapter{Userland Initialization}

\section{Introduction}

\begin{itemize}
    \item \verb+init+
    \item \verb+sbin/rc.init+
    \item \verb+getty+ and \verb+gettytab+
    \item \verb+login+
\end{itemize}
