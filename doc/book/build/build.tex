\part{Build System}

\chapter{Kernel Build}

\verb+kern/Makefile+ is responsible of compiling the kernel and parsing kmod
makefiles. Module makefiles are following the naming convention
\verb+<module>.mk+ and are located under \verb+kern/kmodmakefiles+ directory.

Module makefiles are parsed like normal makefiles but care should be taken when
changing global variables in these makefiles. Module makefiles are mainly
allowed to only append \verb+IDIR+ variable and all other variables should be
more or less specific to the module makefile itself and should begin with the
name of the module.

An example of a module makefile is shown in listing \ref{list:modulemk}.

\lstinputlisting[label=list:modulemk,%
caption=module.mk]{build/module.mk}

The kernel makefile will automatically discover \verb+test-SRC-1+ list and will
compile a new static library based on the compilation units in the list. Name of
the library is derived from the makefile's name and so should be the first word
of the source file list name.


\chapter{Userland Build}

Userland makefiles are constructed from \verb+user_head.mk+,
\verb+user_tail.mk+ and the actual targets between includes. A good example of
a user space makefile is \verb+bin/Makefile+ that compiles tools under
\verb+/bin+. A manifest file is automatically generated by the make system and
it will be used for creating a rootfs image with \verb+tools/mkrootfs.sh+
script.

\chapter{Libraries}
\chapter{Applications}
\chapter{Tools Build}
