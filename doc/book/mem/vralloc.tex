\chapter{vralloc}

\acf{vralloc} is a memory allocator used to allocate blocks of memory
that can be mapped in virtual address space of a user processes or used for
file system buffering. Vralloc implements the \verb+geteblk()+ part of the interface
described by \verb+buf.h+, meaning allocating memory for generic use.

\verb+vreg+ struct is the intrenal representation of a generic page aligned
allocation made from dynmem and \verb+struct buf+ is the external interface
used to pass allocated memory for external users.

\begin{figure}
\begin{verbatim}
                      last_vreg
                               \
+----------------+     +-----------------+     +-------+
| .paddr = 0x100 |     | .paddr = 0x1500 |     |       |
| .count = 100   |<--->| .count = 20     |<--->|       |
| .map[]         |     | .map[]          |     |       |
+----------------+     +-----------------+     +-------+
\end{verbatim}
\caption{vregion blocks allocated from dynmem.}
\label{figure:vralloc_blocks}
\end{figure}

\begin{figure}
\begin{verbatim}
+-----------------+     +-----------+
| struct buf b    |  -> |  vreg     |
+-----------------+  |  +-----------+
| ...             |  |  | .paddr    |
| .allocator_data | --  | .count    |
|                 |     | .map[]    |
+-----------------+     +-----------+
\end{verbatim}
\caption{vralloc and buffer interface.}
\label{figure:vrregbufapi}
\end{figure}
